\documentclass[12pt,a4paper]{scrartcl}
\usepackage[utf8x]{inputenc}
\usepackage{ucs}
\usepackage{amsmath}
\usepackage{amsfonts}
\usepackage{amssymb}
\usepackage{rotating}
\usepackage{url}
\usepackage[ngerman]{babel}
\usepackage{listings}
%% Seitenrändereinstellungen!!!!!!!!!! -> WICHTIG für coole Texte
\usepackage[a4paper , lmargin = {2.5cm} , rmargin = {2.5 cm} , tmargin = {2.5cm} , bmargin = {2.5cm} ]{geometry}

\usepackage{psfrag,graphicx}

%\usepackage{hyperref} %man kann drauf klicken um zu referenz zu gelangen
%KEYWORDS
\newenvironment{keywords}%
   {\begin{trivlist}\item[]{\bfseries\sffamily Keywords:}\ }
   {\end{trivlist}}


\begin{document}


\title{Unternehmensorientierung\\
	  Buisenessplan\\
	  Der Super Gerät}
\author{Vitalij Stepanov, Andreas Dubs, Friedrich Groß}
\date{\today}
\maketitle


%\tableofcontents
\newpage
\pagestyle{myheadings}
\markright{Unternehmensorientierung - Buisenessplan - Der Super Gerät - \today}
 
\section{Management-Zusammenfassung}

\section{Unternehmensbeschreibung}

\section{Führungsteam und Organisation}
Das Führungsteam besteht aus drei Informatik Studenten, von denen jeder vor dem Studium eine Ausbildung gemacht hat.

Friedrich Groß ist 26 Jahre alt und hat eine Ausbildung zum Mechatroniker (Industrie) gemacht.
In seiner Ausbildung hat er Fachwissen aus dem Bereich Elektrotechnik, Mechanik und SPS- und Roboterprogrammierung erlernt.

Andreas Dubs ist 30 Jahre als und hat eine Ausbildung zum Elektrotechniker Gebäudetechnik gemacht.
In seiner Ausbildung hat er Fachwissen aus dem Bereich Elektrotechnik, Fotovoltaikanlagen und Siemens-Logo programmierung erlernt

Vitalij Stepanov ist 35 Jahre alt und hat eine Ausbildung zum Koch und Fachinformatiker gemacht.
Als Fachinformatiker hat er diverse Programmiersprachen erlernt und als Koch das Wissen darüber wie man fachgerecht Essen zubereitet.

Vereinigmant das Fachwissen erhält man großes Spektrum an Wissen, was bei der Entwicklung und Herstellung eines
automatisierten 

Dieses Fachwissen kann er bei der Entwicklung und Umsetzung des Automaten verwenden.

\section{Produkt und Dienstleistung}

\section{Markt und Wettbewerb}

\section{Entwicklung und Produktion}

\section{Einkauf und Logistik}

\section{Einkauf und Logistik}

%\nocite{1046015}
%\bibliography{literatur}{}
%\bibliographystyle{plain} %unterstützt kein URL feld
%\bibliographystyle{plaindin}  %immerhin urls aber deutsch..
%\bibliographystyle{Natbib}

\end{document}
