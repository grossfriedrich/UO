
\section{Produkte und Dienstleistungen}

\subsection{Produktbeschreibung} 
Ein innovatives, vollautomatisches Donergerät ist in der Lage, Döner ohne
Einsatz von Personal nach Kundenwunsch herzustellen und abzurechnen. Der Kunde
bestimmt über ein Touchscreen die Inhalte des Gerichtes, sowie wählt ein Getränk
wie CocaCola, Fanta usw. aus. Anschließend wird der Benutzer aufgefordert mit
einer EC-Karte oder Bar den Auftrag zu bezahlen. Nach der Bezahlung bekommt der
Kunde die Abholmarke mit der eindeutigen Nummer und der Auftrag wird in die
Warteschlange eingestellt. Über die Glassscheibe kann der Kunde die Zubereitung
live verfolgen. Nachdem der Döner fertiggestellt wurde, wird die Bestellnummer
angezeigt und der Kunde wird aufgefordert die Marke in das Automat einzuführen,
un anschließend das bestellte Produkt mit dem Getränk zu bekommen. Das Getränk
wird in das Einwegbecher eingeschenkt.
\subsection{Technische Merkmale}
Der Dönerautomat benötigt, im Vergleich zu einem Dönerladen, einen geringeren
Platzbedarf. Das Gerät hat die die Maße von 1,70m x 2,3m x 1,0m und benötigt den
Zugang zum Befühlen von Ingredienten. Für die Herstellung werden Rohstofe
benutzt, die für die Herstellung von Speisen geeignet sind. Ein 15 Zoll
Touchscreen und die intuitive Bedienoberfläche erleichtert die Eingabe der
Aufträge. Mittels einer App wird der Techniker über Fühlstände bzw. aufgetretene
Störungen informiert und kann die Fehlerursache beseitigen. Dadurch kann ein
Techniker mehrere Automaten instandhalten. Einmal täglich wird der Automat
gereinigt, um den Regeln der Gastronomie gerecht zu werden.
\subsection{Konkurrenzprodukte}
Derzeit existieren keine deratrigen Produkte auf dem Markt, die Zubereitung wird
herkömmlich von dem Gastwirt zubereitet. Die Innovation des Produkts verspricht
den großen Abstand von der Konkurrenz.
\subsection{Vorteile vs. Nachteile}
Nachfolgend werden die Vorteile den Nachteilen gegenübergestellt.
\begin{itemize}
  \item[$+$] geringerer Platzbedarf
  \item[$+$] innovativ
  \item[$+$] kürzere Wartezeit
  \item[$+$] besserer Durchsatz
  \item[$+$] mindert die Lohnkosten
  \item[$+$] billigere Dönerpreise
  \item[$+$] Reduziert den Körperkontakt bei Zubereitung\\
  \item[$-$] teuere Anschaffung
  \item[$-$] kein Personenkontakt
\end{itemize}
Aus dieser Liste ist deutlich zu sehen, dass das Produkt ein sehr gutes Potential hat
und trotz hohen Anschaffungspreis sich durchsetzen kann
\subsection{Zielgruppen}
\begin{itemize}
	\item Die Enverbraucher sind diejenigen Menschen, die in der Eile sind und was schnell
		unterwegs essen wollen, anderseits aber ein hohes Wert auf die frische der
		Produkte legen.
	\item Die Berteiber sind juristische bzw. natürliche Personen, die mittels des
		vollautomatischen Dönergerätes die Leistungen an Endverbraucher anbieten wollen.
	\item Als Installationsorte eignen sich herforragend die Bahnhöfe, Einkaufszentren,
		Hochschulen bzw. sehr große Betriebe, die in den Mittagspausen den Mitarbeitern 
		die Wartezeiten verkürzen wollen.
\end{itemize}




\subsection{Kundennutzen}
Durch die Unterbrechungslose Abarbeitung reduziert sich die Wartezeit auf maximal 1 min pro Dönner.
